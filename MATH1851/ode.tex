\documentclass{scrartcl}

\usepackage{graphicx}
\usepackage{amsmath}
\usepackage{unicode-math}
\usepackage{siunitx}

\title{MATH1851 Part 2: Ordinary Differential Equations}
\author{James Sungarda}
\date{9 April 2025}

\graphicspath{{images/}}


\begin{document}
\maketitle

\section{Integrating Factor}
Given an equation of the form:
\[
\frac{dy}{dx} + p(x)y = q(x)
\] 
The integrating factor is given by:
\[
I = e^{\int p(x) \, dx}
\]
Multiplying the entire equation by \(I\) gives \(I\frac{dy}{dx} + Ip(x)y = Iq(x)\), which can be used to solve the ODE:

\begin{gather}
    \frac{d}{dx}(Iy) = Iq(x) \\
    Iy = \int Iq(x) \, dx + C  \\
    y = \frac{1}{I}\left(\int Iq(x) \, dx + C\right)
\end{gather}

\section{Homogeneous Linear ODE}
Given an equation of the form \(\frac{dy}{dx} + p(x)y = 0\)

Let \(w = \frac{y}{x}\), and substitute \(y = wx \implies \frac{dy}{dx} = x\frac{dw}{dx} + w\) into the equation, the equation can then be turned into a solvable separable linear ODE.

\section{Bernoulli's equation}
A Bernoulli equation is of the form:
\[
\frac{dy}{dx} + p(x)y = q(x)y^n
\]
where \(n \neq 0, 1\).
Use the substitution:
\[
u = y^{1-n} \iff y = u^{\frac{1}{1-n}}
\]
The equation will then become a solvable ODE via integrating factor.

\section{Exact ODE}
An equation of the form:
\[
M(x,y)dx + N(x,y)dy = 0
\]
is exact if:
\[
\frac{\partial M}{\partial y} = \frac{\partial N}{\partial x}
\]
If the equation is exact, then:

\begin{gather}
\frac{\partial F}{\partial x} = M \quad \text{and} \quad \frac{\partial F}{\partial y} = N \\
F = \int M \, dx + g(y) \\
F = \int N \, dy + h(x)
\end{gather}

where \(F(x,y)\) is the solution to the ODE. To find \(g(y)\) after finding \(h(x)\):

\begin{center}
    \begin{gather}
            \frac{\partial F}{\partial y} = \frac{\partial}{\partial y}\left(\int M \, dx + g(y)\right) \\
            \therefore{} \frac{\partial F}{\partial y} = N
    \end{gather}
\end{center}

\section{Cauchy-Euler equations}
A Cauchy-Euler equation is of the form:
\[
x^2\frac{d^2y}{dx^2} + a x \frac{dy}{dx} + b y = 0
\]
where \(a\) and \(b\) are constants and there are no other constants. 

Let \(y = x^\lambda\), then solve it as a quadratic equation of sorts.

General solution if:

\[
\lambda_1 \neq \lambda_2 \implies y = C_1 x^{\lambda_1} + C_2 x^{\lambda_2}
\]
\[
\lambda_1 = \lambda_2 \implies y = C_1 x^{\lambda_1} + C_2 x^{\lambda_1} \ln(x)
\]
If \(\lambda_1\) and \(\lambda_2\) are complex, then the general solution is:
\[
y = x^{\alpha} \left( C_1 \cos(\beta \ln(x)) + C_2 \sin(\beta \ln(x)) \right)
\]
where \(\lambda_{1,2} = \alpha \pm i\beta\).

\section{Second order linear equations}
A second order linear equation is of the form:
\[
a\frac{d^2y}{dx^2} + b\frac{dy}{dx} + cy = 0
\]
where \(a\), \(b\), and \(c\) are constants.
Substitute \(y = e^{\lambda x}\), and solve it as a normal quadratic equation, since \(e^{\lambda x} \neq 0\).

The general solution if:

\[
\lambda_1 \neq \lambda_2 \implies y = C_1 e^{\lambda_1 x} + C_2 e^{\lambda_2 x}
\]
\[
\lambda_1 = \lambda_2 \implies y = C_1 e^{\lambda_1 x} + C_2 x e^{\lambda_1 x}
\]

If \(\lambda_1\) and \(\lambda_2\) are complex, then the general solution is:
\[
y = e^{\alpha x} \left( C_1 \cos(\beta x) + C_2 \sin(\beta x) \right)
\]
where \(\lambda_{1,2} = \alpha \pm i\beta\).

\subsection{Non-homogeneous second order linear equations via undetermined coefficients}
Guesses list (table):
\begin{center}
    \begin{tabular}{|c|c|}
        \hline
        Form of \(g(x)\) & Guess \\
        \hline
        \(e^{ax}\) & \(Ae^{ax}\) \\
        \hline
        \(\sin(bx)\) & \(A\sin(bx) + B\cos(bx)\) \\
        \hline
        \(\cos(bx)\) & \(A\sin(bx) + B\cos(bx)\) \\
        \hline
        \(x^n\) & \(Ax^n + Bx^{n-1} + Cx^{n-2} + ...\) \\
        \hline
    \end{tabular}
\end{center}
If the guess is a solution to the homogeneous equation, then multiply the guess by \(x^k\) where \(k\) is the smallest integer such that \(x^k\) times the guess is not a solution to the homogeneous equation.

\subsection{D-Operator}
D-operator is defined as \(D = \frac{d}{dx}\), where \(x\) is the variable.

If RHS is \(f(x)\), then the D-operator to use is:
\[
A_0x^n + A_1x^{n-1} + ... + A_n \implies L = D^{n+1}
\]
\[
A_0x^ne^{ax} + A_1x^{n-1}e^{ax} + ... + A_n e^{ax} \implies L = (D-a)^{n+1}
\]
\[
A_0x^ne^{ax}\sin(bx) + A_0x^{n}e^{ax}\cos(bx) + ... + A_n e^{ax} \cos(bx) \implies L = ((D-a)^2 + b^2)^{n+1}
\]

\section{Riccati's equation}
A Riccati equation is of the form:
\[
\frac{dy}{dx} = p(x)y^2 + q(x)y + r(x)
\]
where \(p(x)\), \(q(x)\), and \(r(x)\) are functions of \(x\).

Do an educated guess to find a solution \(u(x)\), and substitute it to:
\[
    Y(x) = u + \frac{1}{v}
\]
where \(v(x)\) is a function of \(x\). The equation will then become a linear ODE.

A shortcut to use is:
\[
\frac{dv}{dx} + (2p(x)u(x) + q(x))v = -p(x)
\]

\section{When to use each method}
Given a differential equation consisting of the function \(f(x) = y\), and one variable \(x\):

If \(f(x)\) is a \textit{first order} differential equation:
\begin{itemize}
    \item Use \textbf{Integrating Factor} if the equation is linear (highest order of \(y\) is 1).
    \item Use \textbf{Homogeneous Linear ODE} if the equation is homogeneous.
    \item Use \textbf{Bernoulli's equation} if the equation is of the form \(\frac{dy}{dx} + p(x)y = q(x)y^n\).
    \item Use \textbf{Riccati's equation} if the equation is of the form \(\frac{dy}{dx} = p(x) + q(x)y + r(x)y^2\).
    \item Use \textbf{Exact ODE} if the equation is exact.
\end{itemize}

If \(f(x)\) is a \textit{higher order} differential equation:
\begin{itemize}
    \item Use \textbf{Cauchy-Euler equations} if the equation \textbf{contains another variable \(x\)} and is of the form \(x^2\frac{d^2y}{dx^2} + a x \frac{dy}{dx} + b y = 0\).
    \item Use \textbf{Second order linear equations} if the equation is of the form \(a\frac{d^2y}{dx^2} + b\frac{dy}{dx} + cy = 0\), where \(a\), \(b\), and \(c\) are constants.
\end{itemize}


\section{Variation of Parameters}
Given two known solutions \(y_1\) and \(y_2\), the general solutions can be found with:
Here is the proof


\section{Extras}
\subsection{Riccati shortcut proof}
Given a differential equation \(\frac{dy}{dx} = p(x)y^2 + q(x)y + r(x)\), and \(u(x)\) is a known solution:
Create a substitution \(Y(x) = u(x) + \frac{1}{v(x)}\):
\begin{gather}
    Y(x) = u + \frac{1}{v} \implies \frac{dY}{dx} = \frac{du}{dx} - \frac{1}{v^2}\frac{dv}{dx} \\
    \frac{du}{dx} - \frac{1}{v^2}\frac{dv}{dx} = p\left(u^2+\frac{2u}{v} + \frac{1}{v^2}\right) + q\left(u + \frac{1}{v}\right) + r \\
    \frac{du}{dx}v^2 - \frac{dv}{dx} = (pu^2 + qu + r)v^2 + (2pu + q)v + p \\
    \frac{dv}{dx} + (2pu+q)v = (\frac{du}{dx} - (pu^2 + qu + r))v^2 - p \\
    \therefore{} \frac{dv}{dx} + (2pu + q)v = -p
\end{gather}

Applying these principles to our problem:
Let $F(s) = \frac{1}{s^2+b^2}$, so $f(t) = \frac{1}{b}\sin(bt)$
Then, using the time-shift property with $a = 2$:

% $\mathcal{L}^{-1}\left{\frac{e^{-2s}}{s^2+b^2}\right} = \frac{1}{b}\sin(b(t-2))u(t-2)$

\end{document}
